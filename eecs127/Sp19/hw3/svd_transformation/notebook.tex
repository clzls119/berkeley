
% Default to the notebook output style

    


% Inherit from the specified cell style.




    
\documentclass[11pt]{article}

    
    
    \usepackage[T1]{fontenc}
    % Nicer default font (+ math font) than Computer Modern for most use cases
    \usepackage{mathpazo}

    % Basic figure setup, for now with no caption control since it's done
    % automatically by Pandoc (which extracts ![](path) syntax from Markdown).
    \usepackage{graphicx}
    % We will generate all images so they have a width \maxwidth. This means
    % that they will get their normal width if they fit onto the page, but
    % are scaled down if they would overflow the margins.
    \makeatletter
    \def\maxwidth{\ifdim\Gin@nat@width>\linewidth\linewidth
    \else\Gin@nat@width\fi}
    \makeatother
    \let\Oldincludegraphics\includegraphics
    % Set max figure width to be 80% of text width, for now hardcoded.
    \renewcommand{\includegraphics}[1]{\Oldincludegraphics[width=.8\maxwidth]{#1}}
    % Ensure that by default, figures have no caption (until we provide a
    % proper Figure object with a Caption API and a way to capture that
    % in the conversion process - todo).
    \usepackage{caption}
    \DeclareCaptionLabelFormat{nolabel}{}
    \captionsetup{labelformat=nolabel}

    \usepackage{adjustbox} % Used to constrain images to a maximum size 
    \usepackage{xcolor} % Allow colors to be defined
    \usepackage{enumerate} % Needed for markdown enumerations to work
    \usepackage{geometry} % Used to adjust the document margins
    \usepackage{amsmath} % Equations
    \usepackage{amssymb} % Equations
    \usepackage{textcomp} % defines textquotesingle
    % Hack from http://tex.stackexchange.com/a/47451/13684:
    \AtBeginDocument{%
        \def\PYZsq{\textquotesingle}% Upright quotes in Pygmentized code
    }
    \usepackage{upquote} % Upright quotes for verbatim code
    \usepackage{eurosym} % defines \euro
    \usepackage[mathletters]{ucs} % Extended unicode (utf-8) support
    \usepackage[utf8x]{inputenc} % Allow utf-8 characters in the tex document
    \usepackage{fancyvrb} % verbatim replacement that allows latex
    \usepackage{grffile} % extends the file name processing of package graphics 
                         % to support a larger range 
    % The hyperref package gives us a pdf with properly built
    % internal navigation ('pdf bookmarks' for the table of contents,
    % internal cross-reference links, web links for URLs, etc.)
    \usepackage{hyperref}
    \usepackage{longtable} % longtable support required by pandoc >1.10
    \usepackage{booktabs}  % table support for pandoc > 1.12.2
    \usepackage[inline]{enumitem} % IRkernel/repr support (it uses the enumerate* environment)
    \usepackage[normalem]{ulem} % ulem is needed to support strikethroughs (\sout)
                                % normalem makes italics be italics, not underlines
    

    
    
    % Colors for the hyperref package
    \definecolor{urlcolor}{rgb}{0,.145,.698}
    \definecolor{linkcolor}{rgb}{.71,0.21,0.01}
    \definecolor{citecolor}{rgb}{.12,.54,.11}

    % ANSI colors
    \definecolor{ansi-black}{HTML}{3E424D}
    \definecolor{ansi-black-intense}{HTML}{282C36}
    \definecolor{ansi-red}{HTML}{E75C58}
    \definecolor{ansi-red-intense}{HTML}{B22B31}
    \definecolor{ansi-green}{HTML}{00A250}
    \definecolor{ansi-green-intense}{HTML}{007427}
    \definecolor{ansi-yellow}{HTML}{DDB62B}
    \definecolor{ansi-yellow-intense}{HTML}{B27D12}
    \definecolor{ansi-blue}{HTML}{208FFB}
    \definecolor{ansi-blue-intense}{HTML}{0065CA}
    \definecolor{ansi-magenta}{HTML}{D160C4}
    \definecolor{ansi-magenta-intense}{HTML}{A03196}
    \definecolor{ansi-cyan}{HTML}{60C6C8}
    \definecolor{ansi-cyan-intense}{HTML}{258F8F}
    \definecolor{ansi-white}{HTML}{C5C1B4}
    \definecolor{ansi-white-intense}{HTML}{A1A6B2}

    % commands and environments needed by pandoc snippets
    % extracted from the output of `pandoc -s`
    \providecommand{\tightlist}{%
      \setlength{\itemsep}{0pt}\setlength{\parskip}{0pt}}
    \DefineVerbatimEnvironment{Highlighting}{Verbatim}{commandchars=\\\{\}}
    % Add ',fontsize=\small' for more characters per line
    \newenvironment{Shaded}{}{}
    \newcommand{\KeywordTok}[1]{\textcolor[rgb]{0.00,0.44,0.13}{\textbf{{#1}}}}
    \newcommand{\DataTypeTok}[1]{\textcolor[rgb]{0.56,0.13,0.00}{{#1}}}
    \newcommand{\DecValTok}[1]{\textcolor[rgb]{0.25,0.63,0.44}{{#1}}}
    \newcommand{\BaseNTok}[1]{\textcolor[rgb]{0.25,0.63,0.44}{{#1}}}
    \newcommand{\FloatTok}[1]{\textcolor[rgb]{0.25,0.63,0.44}{{#1}}}
    \newcommand{\CharTok}[1]{\textcolor[rgb]{0.25,0.44,0.63}{{#1}}}
    \newcommand{\StringTok}[1]{\textcolor[rgb]{0.25,0.44,0.63}{{#1}}}
    \newcommand{\CommentTok}[1]{\textcolor[rgb]{0.38,0.63,0.69}{\textit{{#1}}}}
    \newcommand{\OtherTok}[1]{\textcolor[rgb]{0.00,0.44,0.13}{{#1}}}
    \newcommand{\AlertTok}[1]{\textcolor[rgb]{1.00,0.00,0.00}{\textbf{{#1}}}}
    \newcommand{\FunctionTok}[1]{\textcolor[rgb]{0.02,0.16,0.49}{{#1}}}
    \newcommand{\RegionMarkerTok}[1]{{#1}}
    \newcommand{\ErrorTok}[1]{\textcolor[rgb]{1.00,0.00,0.00}{\textbf{{#1}}}}
    \newcommand{\NormalTok}[1]{{#1}}
    
    % Additional commands for more recent versions of Pandoc
    \newcommand{\ConstantTok}[1]{\textcolor[rgb]{0.53,0.00,0.00}{{#1}}}
    \newcommand{\SpecialCharTok}[1]{\textcolor[rgb]{0.25,0.44,0.63}{{#1}}}
    \newcommand{\VerbatimStringTok}[1]{\textcolor[rgb]{0.25,0.44,0.63}{{#1}}}
    \newcommand{\SpecialStringTok}[1]{\textcolor[rgb]{0.73,0.40,0.53}{{#1}}}
    \newcommand{\ImportTok}[1]{{#1}}
    \newcommand{\DocumentationTok}[1]{\textcolor[rgb]{0.73,0.13,0.13}{\textit{{#1}}}}
    \newcommand{\AnnotationTok}[1]{\textcolor[rgb]{0.38,0.63,0.69}{\textbf{\textit{{#1}}}}}
    \newcommand{\CommentVarTok}[1]{\textcolor[rgb]{0.38,0.63,0.69}{\textbf{\textit{{#1}}}}}
    \newcommand{\VariableTok}[1]{\textcolor[rgb]{0.10,0.09,0.49}{{#1}}}
    \newcommand{\ControlFlowTok}[1]{\textcolor[rgb]{0.00,0.44,0.13}{\textbf{{#1}}}}
    \newcommand{\OperatorTok}[1]{\textcolor[rgb]{0.40,0.40,0.40}{{#1}}}
    \newcommand{\BuiltInTok}[1]{{#1}}
    \newcommand{\ExtensionTok}[1]{{#1}}
    \newcommand{\PreprocessorTok}[1]{\textcolor[rgb]{0.74,0.48,0.00}{{#1}}}
    \newcommand{\AttributeTok}[1]{\textcolor[rgb]{0.49,0.56,0.16}{{#1}}}
    \newcommand{\InformationTok}[1]{\textcolor[rgb]{0.38,0.63,0.69}{\textbf{\textit{{#1}}}}}
    \newcommand{\WarningTok}[1]{\textcolor[rgb]{0.38,0.63,0.69}{\textbf{\textit{{#1}}}}}
    
    
    % Define a nice break command that doesn't care if a line doesn't already
    % exist.
    \def\br{\hspace*{\fill} \\* }
    % Math Jax compatability definitions
    \def\gt{>}
    \def\lt{<}
    % Document parameters
    \title{svd\_transformation}
    
    
    

    % Pygments definitions
    
\makeatletter
\def\PY@reset{\let\PY@it=\relax \let\PY@bf=\relax%
    \let\PY@ul=\relax \let\PY@tc=\relax%
    \let\PY@bc=\relax \let\PY@ff=\relax}
\def\PY@tok#1{\csname PY@tok@#1\endcsname}
\def\PY@toks#1+{\ifx\relax#1\empty\else%
    \PY@tok{#1}\expandafter\PY@toks\fi}
\def\PY@do#1{\PY@bc{\PY@tc{\PY@ul{%
    \PY@it{\PY@bf{\PY@ff{#1}}}}}}}
\def\PY#1#2{\PY@reset\PY@toks#1+\relax+\PY@do{#2}}

\expandafter\def\csname PY@tok@w\endcsname{\def\PY@tc##1{\textcolor[rgb]{0.73,0.73,0.73}{##1}}}
\expandafter\def\csname PY@tok@c\endcsname{\let\PY@it=\textit\def\PY@tc##1{\textcolor[rgb]{0.25,0.50,0.50}{##1}}}
\expandafter\def\csname PY@tok@cp\endcsname{\def\PY@tc##1{\textcolor[rgb]{0.74,0.48,0.00}{##1}}}
\expandafter\def\csname PY@tok@k\endcsname{\let\PY@bf=\textbf\def\PY@tc##1{\textcolor[rgb]{0.00,0.50,0.00}{##1}}}
\expandafter\def\csname PY@tok@kp\endcsname{\def\PY@tc##1{\textcolor[rgb]{0.00,0.50,0.00}{##1}}}
\expandafter\def\csname PY@tok@kt\endcsname{\def\PY@tc##1{\textcolor[rgb]{0.69,0.00,0.25}{##1}}}
\expandafter\def\csname PY@tok@o\endcsname{\def\PY@tc##1{\textcolor[rgb]{0.40,0.40,0.40}{##1}}}
\expandafter\def\csname PY@tok@ow\endcsname{\let\PY@bf=\textbf\def\PY@tc##1{\textcolor[rgb]{0.67,0.13,1.00}{##1}}}
\expandafter\def\csname PY@tok@nb\endcsname{\def\PY@tc##1{\textcolor[rgb]{0.00,0.50,0.00}{##1}}}
\expandafter\def\csname PY@tok@nf\endcsname{\def\PY@tc##1{\textcolor[rgb]{0.00,0.00,1.00}{##1}}}
\expandafter\def\csname PY@tok@nc\endcsname{\let\PY@bf=\textbf\def\PY@tc##1{\textcolor[rgb]{0.00,0.00,1.00}{##1}}}
\expandafter\def\csname PY@tok@nn\endcsname{\let\PY@bf=\textbf\def\PY@tc##1{\textcolor[rgb]{0.00,0.00,1.00}{##1}}}
\expandafter\def\csname PY@tok@ne\endcsname{\let\PY@bf=\textbf\def\PY@tc##1{\textcolor[rgb]{0.82,0.25,0.23}{##1}}}
\expandafter\def\csname PY@tok@nv\endcsname{\def\PY@tc##1{\textcolor[rgb]{0.10,0.09,0.49}{##1}}}
\expandafter\def\csname PY@tok@no\endcsname{\def\PY@tc##1{\textcolor[rgb]{0.53,0.00,0.00}{##1}}}
\expandafter\def\csname PY@tok@nl\endcsname{\def\PY@tc##1{\textcolor[rgb]{0.63,0.63,0.00}{##1}}}
\expandafter\def\csname PY@tok@ni\endcsname{\let\PY@bf=\textbf\def\PY@tc##1{\textcolor[rgb]{0.60,0.60,0.60}{##1}}}
\expandafter\def\csname PY@tok@na\endcsname{\def\PY@tc##1{\textcolor[rgb]{0.49,0.56,0.16}{##1}}}
\expandafter\def\csname PY@tok@nt\endcsname{\let\PY@bf=\textbf\def\PY@tc##1{\textcolor[rgb]{0.00,0.50,0.00}{##1}}}
\expandafter\def\csname PY@tok@nd\endcsname{\def\PY@tc##1{\textcolor[rgb]{0.67,0.13,1.00}{##1}}}
\expandafter\def\csname PY@tok@s\endcsname{\def\PY@tc##1{\textcolor[rgb]{0.73,0.13,0.13}{##1}}}
\expandafter\def\csname PY@tok@sd\endcsname{\let\PY@it=\textit\def\PY@tc##1{\textcolor[rgb]{0.73,0.13,0.13}{##1}}}
\expandafter\def\csname PY@tok@si\endcsname{\let\PY@bf=\textbf\def\PY@tc##1{\textcolor[rgb]{0.73,0.40,0.53}{##1}}}
\expandafter\def\csname PY@tok@se\endcsname{\let\PY@bf=\textbf\def\PY@tc##1{\textcolor[rgb]{0.73,0.40,0.13}{##1}}}
\expandafter\def\csname PY@tok@sr\endcsname{\def\PY@tc##1{\textcolor[rgb]{0.73,0.40,0.53}{##1}}}
\expandafter\def\csname PY@tok@ss\endcsname{\def\PY@tc##1{\textcolor[rgb]{0.10,0.09,0.49}{##1}}}
\expandafter\def\csname PY@tok@sx\endcsname{\def\PY@tc##1{\textcolor[rgb]{0.00,0.50,0.00}{##1}}}
\expandafter\def\csname PY@tok@m\endcsname{\def\PY@tc##1{\textcolor[rgb]{0.40,0.40,0.40}{##1}}}
\expandafter\def\csname PY@tok@gh\endcsname{\let\PY@bf=\textbf\def\PY@tc##1{\textcolor[rgb]{0.00,0.00,0.50}{##1}}}
\expandafter\def\csname PY@tok@gu\endcsname{\let\PY@bf=\textbf\def\PY@tc##1{\textcolor[rgb]{0.50,0.00,0.50}{##1}}}
\expandafter\def\csname PY@tok@gd\endcsname{\def\PY@tc##1{\textcolor[rgb]{0.63,0.00,0.00}{##1}}}
\expandafter\def\csname PY@tok@gi\endcsname{\def\PY@tc##1{\textcolor[rgb]{0.00,0.63,0.00}{##1}}}
\expandafter\def\csname PY@tok@gr\endcsname{\def\PY@tc##1{\textcolor[rgb]{1.00,0.00,0.00}{##1}}}
\expandafter\def\csname PY@tok@ge\endcsname{\let\PY@it=\textit}
\expandafter\def\csname PY@tok@gs\endcsname{\let\PY@bf=\textbf}
\expandafter\def\csname PY@tok@gp\endcsname{\let\PY@bf=\textbf\def\PY@tc##1{\textcolor[rgb]{0.00,0.00,0.50}{##1}}}
\expandafter\def\csname PY@tok@go\endcsname{\def\PY@tc##1{\textcolor[rgb]{0.53,0.53,0.53}{##1}}}
\expandafter\def\csname PY@tok@gt\endcsname{\def\PY@tc##1{\textcolor[rgb]{0.00,0.27,0.87}{##1}}}
\expandafter\def\csname PY@tok@err\endcsname{\def\PY@bc##1{\setlength{\fboxsep}{0pt}\fcolorbox[rgb]{1.00,0.00,0.00}{1,1,1}{\strut ##1}}}
\expandafter\def\csname PY@tok@kc\endcsname{\let\PY@bf=\textbf\def\PY@tc##1{\textcolor[rgb]{0.00,0.50,0.00}{##1}}}
\expandafter\def\csname PY@tok@kd\endcsname{\let\PY@bf=\textbf\def\PY@tc##1{\textcolor[rgb]{0.00,0.50,0.00}{##1}}}
\expandafter\def\csname PY@tok@kn\endcsname{\let\PY@bf=\textbf\def\PY@tc##1{\textcolor[rgb]{0.00,0.50,0.00}{##1}}}
\expandafter\def\csname PY@tok@kr\endcsname{\let\PY@bf=\textbf\def\PY@tc##1{\textcolor[rgb]{0.00,0.50,0.00}{##1}}}
\expandafter\def\csname PY@tok@bp\endcsname{\def\PY@tc##1{\textcolor[rgb]{0.00,0.50,0.00}{##1}}}
\expandafter\def\csname PY@tok@fm\endcsname{\def\PY@tc##1{\textcolor[rgb]{0.00,0.00,1.00}{##1}}}
\expandafter\def\csname PY@tok@vc\endcsname{\def\PY@tc##1{\textcolor[rgb]{0.10,0.09,0.49}{##1}}}
\expandafter\def\csname PY@tok@vg\endcsname{\def\PY@tc##1{\textcolor[rgb]{0.10,0.09,0.49}{##1}}}
\expandafter\def\csname PY@tok@vi\endcsname{\def\PY@tc##1{\textcolor[rgb]{0.10,0.09,0.49}{##1}}}
\expandafter\def\csname PY@tok@vm\endcsname{\def\PY@tc##1{\textcolor[rgb]{0.10,0.09,0.49}{##1}}}
\expandafter\def\csname PY@tok@sa\endcsname{\def\PY@tc##1{\textcolor[rgb]{0.73,0.13,0.13}{##1}}}
\expandafter\def\csname PY@tok@sb\endcsname{\def\PY@tc##1{\textcolor[rgb]{0.73,0.13,0.13}{##1}}}
\expandafter\def\csname PY@tok@sc\endcsname{\def\PY@tc##1{\textcolor[rgb]{0.73,0.13,0.13}{##1}}}
\expandafter\def\csname PY@tok@dl\endcsname{\def\PY@tc##1{\textcolor[rgb]{0.73,0.13,0.13}{##1}}}
\expandafter\def\csname PY@tok@s2\endcsname{\def\PY@tc##1{\textcolor[rgb]{0.73,0.13,0.13}{##1}}}
\expandafter\def\csname PY@tok@sh\endcsname{\def\PY@tc##1{\textcolor[rgb]{0.73,0.13,0.13}{##1}}}
\expandafter\def\csname PY@tok@s1\endcsname{\def\PY@tc##1{\textcolor[rgb]{0.73,0.13,0.13}{##1}}}
\expandafter\def\csname PY@tok@mb\endcsname{\def\PY@tc##1{\textcolor[rgb]{0.40,0.40,0.40}{##1}}}
\expandafter\def\csname PY@tok@mf\endcsname{\def\PY@tc##1{\textcolor[rgb]{0.40,0.40,0.40}{##1}}}
\expandafter\def\csname PY@tok@mh\endcsname{\def\PY@tc##1{\textcolor[rgb]{0.40,0.40,0.40}{##1}}}
\expandafter\def\csname PY@tok@mi\endcsname{\def\PY@tc##1{\textcolor[rgb]{0.40,0.40,0.40}{##1}}}
\expandafter\def\csname PY@tok@il\endcsname{\def\PY@tc##1{\textcolor[rgb]{0.40,0.40,0.40}{##1}}}
\expandafter\def\csname PY@tok@mo\endcsname{\def\PY@tc##1{\textcolor[rgb]{0.40,0.40,0.40}{##1}}}
\expandafter\def\csname PY@tok@ch\endcsname{\let\PY@it=\textit\def\PY@tc##1{\textcolor[rgb]{0.25,0.50,0.50}{##1}}}
\expandafter\def\csname PY@tok@cm\endcsname{\let\PY@it=\textit\def\PY@tc##1{\textcolor[rgb]{0.25,0.50,0.50}{##1}}}
\expandafter\def\csname PY@tok@cpf\endcsname{\let\PY@it=\textit\def\PY@tc##1{\textcolor[rgb]{0.25,0.50,0.50}{##1}}}
\expandafter\def\csname PY@tok@c1\endcsname{\let\PY@it=\textit\def\PY@tc##1{\textcolor[rgb]{0.25,0.50,0.50}{##1}}}
\expandafter\def\csname PY@tok@cs\endcsname{\let\PY@it=\textit\def\PY@tc##1{\textcolor[rgb]{0.25,0.50,0.50}{##1}}}

\def\PYZbs{\char`\\}
\def\PYZus{\char`\_}
\def\PYZob{\char`\{}
\def\PYZcb{\char`\}}
\def\PYZca{\char`\^}
\def\PYZam{\char`\&}
\def\PYZlt{\char`\<}
\def\PYZgt{\char`\>}
\def\PYZsh{\char`\#}
\def\PYZpc{\char`\%}
\def\PYZdl{\char`\$}
\def\PYZhy{\char`\-}
\def\PYZsq{\char`\'}
\def\PYZdq{\char`\"}
\def\PYZti{\char`\~}
% for compatibility with earlier versions
\def\PYZat{@}
\def\PYZlb{[}
\def\PYZrb{]}
\makeatother


    % Exact colors from NB
    \definecolor{incolor}{rgb}{0.0, 0.0, 0.5}
    \definecolor{outcolor}{rgb}{0.545, 0.0, 0.0}



    
    % Prevent overflowing lines due to hard-to-break entities
    \sloppy 
    % Setup hyperref package
    \hypersetup{
      breaklinks=true,  % so long urls are correctly broken across lines
      colorlinks=true,
      urlcolor=urlcolor,
      linkcolor=linkcolor,
      citecolor=citecolor,
      }
    % Slightly bigger margins than the latex defaults
    
    \geometry{verbose,tmargin=1in,bmargin=1in,lmargin=1in,rmargin=1in}
    
    

    \begin{document}
    
    
    \maketitle
    
    

    
    \hypertarget{readme}{%
\section{Readme}\label{readme}}

\hypertarget{places-where-solutions-are-required-are-marked-with-todo}{%
\subsection{Places where solutions are required are marked with
\#TODO}\label{places-where-solutions-are-required-are-marked-with-todo}}

\hypertarget{you-will-not-need-to-modify-any-section-not-marked-as-todo-to-answer-this-question.}{%
\subsection{You will not need to modify any section not marked as \#TODO
to answer this
question.}\label{you-will-not-need-to-modify-any-section-not-marked-as-todo-to-answer-this-question.}}

\hypertarget{make-sure-the-helper-file.-svd_transformation_helper.py-is-in-the-same-folder-as-this-.ipynb}{%
\subsection{Make sure the helper file. svd\_transformation\_helper.py is
in the same folder as this
.ipynb}\label{make-sure-the-helper-file.-svd_transformation_helper.py-is-in-the-same-folder-as-this-.ipynb}}

\hypertarget{make-sure-you-have-numpy-matplotlib-and-itertools-packages-installed-for-python}{%
\subsection{Make sure you have numpy, matplotlib and itertools packages
installed for
python}\label{make-sure-you-have-numpy-matplotlib-and-itertools-packages-installed-for-python}}

\hypertarget{q3b-has-3-subparts-i-ii-and-iii}{%
\subsection{Q3b has 3 subparts i, ii, and
iii}\label{q3b-has-3-subparts-i-ii-and-iii}}

\hypertarget{q3c-has-4-subparts-i-ii-iii-and-iv}{%
\subsection{Q3c has 4 subparts i, ii, iii and
iv}\label{q3c-has-4-subparts-i-ii-iii-and-iv}}

\hypertarget{q3d-has-2-subparts-iii}{%
\subsection{Q3d has 2 subparts i,ii}\label{q3d-has-2-subparts-iii}}

\hypertarget{q3e-has-only-1-subpart}{%
\subsection{Q3e has only 1 subpart}\label{q3e-has-only-1-subpart}}

    \begin{Verbatim}[commandchars=\\\{\}]
{\color{incolor}In [{\color{incolor}1}]:} \PY{k+kn}{import} \PY{n+nn}{numpy} \PY{k}{as} \PY{n+nn}{np}
        \PY{k+kn}{import} \PY{n+nn}{matplotlib}\PY{n+nn}{.}\PY{n+nn}{pyplot} \PY{k}{as} \PY{n+nn}{plt}
        \PY{o}{\PYZpc{}}\PY{k}{matplotlib} inline
        \PY{k+kn}{from} \PY{n+nn}{svd\PYZus{}transformation\PYZus{}helper} \PY{k}{import} \PY{n}{visualize\PYZus{}function}
        \PY{k+kn}{from} \PY{n+nn}{svd\PYZus{}transformation\PYZus{}helper} \PY{k}{import} \PY{n}{matrix\PYZus{}equals}\PY{p}{,} \PY{n}{is\PYZus{}orthogonal}
\end{Verbatim}


    \begin{Verbatim}[commandchars=\\\{\}]
{\color{incolor}In [{\color{incolor}2}]:} \PY{n}{DISABLE\PYZus{}CHECKS} \PY{o}{=} \PY{k+kc}{False} \PY{c+c1}{\PYZsh{}Set this to True only if you get Value Errors about inputs even }
        \PY{c+c1}{\PYZsh{}when you are sure that what you are inputting is correct.}
        \PY{c+c1}{\PYZsh{}WARNING: Setting this to True and entering wrong inputs can lead to all kinds of crazy results/errors}
        
        
        \PY{k}{def} \PY{n+nf}{visualize}\PY{p}{(}\PY{n}{U} \PY{o}{=} \PY{n}{np}\PY{o}{.}\PY{n}{identity}\PY{p}{(}\PY{l+m+mi}{2}\PY{p}{)}\PY{p}{,} \PY{n}{D} \PY{o}{=} \PY{n}{np}\PY{o}{.}\PY{n}{ones}\PY{p}{(}\PY{l+m+mi}{2}\PY{p}{)}\PY{p}{,} \PY{n}{VT} \PY{o}{=} \PY{n}{np}\PY{o}{.}\PY{n}{identity}\PY{p}{(}\PY{l+m+mi}{2}\PY{p}{)}\PY{p}{,} \PY{n}{num\PYZus{}grid\PYZus{}points\PYZus{}per\PYZus{}dim} \PY{o}{=} \PY{l+m+mi}{200}\PY{p}{,}\PYZbs{}
            \PY{n}{disable\PYZus{}checks} \PY{o}{=} \PY{n}{DISABLE\PYZus{}CHECKS}\PY{p}{,} \PY{n}{show\PYZus{}original} \PY{o}{=} \PY{k+kc}{True}\PY{p}{,} \PY{n}{show\PYZus{}VT} \PY{o}{=} \PY{k+kc}{True}\PY{p}{,} \PY{n}{show\PYZus{}DVT} \PY{o}{=} \PY{k+kc}{True}\PY{p}{,} \PY{n}{show\PYZus{}UDVT} \PY{o}{=} \PY{k+kc}{True}\PY{p}{)}\PY{p}{:}
            \PY{l+s+sd}{\PYZsq{}\PYZsq{}\PYZsq{}}
        \PY{l+s+sd}{    Inputs:}
        \PY{l+s+sd}{    A has singular value decomposition A = U np.diag(D) VT}
        \PY{l+s+sd}{    U: 2 x 2 orthogonal matrix represented as a np.array of shape (2,2)}
        \PY{l+s+sd}{    D: Diagonal entries corresponding to the diagonal matrix in SVD represented as a np.array of shape (2,)}
        \PY{l+s+sd}{    VT: 2 x 2 orthogonal matrix represented as a np.array of shape (2,2)}
        \PY{l+s+sd}{    num\PYZus{}grid\PYZus{}points\PYZus{}per\PYZus{}dim: Spacing of points used to represent circle (Decrease this if plotting is slow)}
        \PY{l+s+sd}{    disable\PYZus{}checks: If False then have checks in  place to make sure dimensions of VT, U are correct, etc. }
        \PY{l+s+sd}{    show\PYZus{}original: If True plots original unit circle and basis vectors}
        \PY{l+s+sd}{    show\PYZus{}VT: If True plots transformation by VT}
        \PY{l+s+sd}{    show\PYZus{}DVT: If True plots transformation by DVT}
        \PY{l+s+sd}{    show\PYZus{}UDVT: If True plots transformation by UDVT}
        \PY{l+s+sd}{    \PYZsq{}\PYZsq{}\PYZsq{}}   
        
            \PY{n}{visualize\PYZus{}function}\PY{p}{(}\PY{n}{U}\PY{o}{=}\PY{n}{U}\PY{p}{,} \PY{n}{D}\PY{o}{=}\PY{n}{D}\PY{p}{,} \PY{n}{VT}\PY{o}{=}\PY{n}{VT}\PY{p}{,} \PY{n}{num\PYZus{}grid\PYZus{}points\PYZus{}per\PYZus{}dim}\PY{o}{=}\PY{n}{num\PYZus{}grid\PYZus{}points\PYZus{}per\PYZus{}dim}\PY{p}{,} \PY{n}{disable\PYZus{}checks}\PY{o}{=}\PY{n}{disable\PYZus{}checks}\PY{p}{,}\PYZbs{}
                             \PY{n}{show\PYZus{}original}\PY{o}{=}\PY{n}{show\PYZus{}original}\PY{p}{,} \PY{n}{show\PYZus{}VT}\PY{o}{=}\PY{n}{show\PYZus{}VT}\PY{p}{,} \PY{n}{show\PYZus{}DVT}\PY{o}{=}\PY{n}{show\PYZus{}DVT}\PY{p}{,} \PY{n}{show\PYZus{}UDVT}\PY{o}{=}\PY{n}{show\PYZus{}UDVT}\PY{p}{)}
            
\end{Verbatim}


    \hypertarget{we-start-by-looking-at-transformation-by-vt-d-u-separately.}{%
\section{\texorpdfstring{We start by looking at transformation by
\(V^T, D, U\)
separately.}{We start by looking at transformation by V\^{}T, D, U separately.}}\label{we-start-by-looking-at-transformation-by-vt-d-u-separately.}}

    \hypertarget{effect-of-the-linear-transformation-by-orthogonal-matrix-vt}{%
\section{\texorpdfstring{Effect of the linear transformation by
orthogonal matrix
\(V^T\)}{Effect of the linear transformation by orthogonal matrix V\^{}T}}\label{effect-of-the-linear-transformation-by-orthogonal-matrix-vt}}

A 2 x 2 orthogonal matrix can be viewed as a linear transformation that
performs some combination of rotations and reflections. Note that both
rotation and reflection are operations that preserve length of vectors
and angle between vectors.

    \hypertarget{vt-as-a-rotation-matrix}{%
\subsection{\texorpdfstring{\(V^T\) as a rotation
matrix}{V\^{}T as a rotation matrix}}\label{vt-as-a-rotation-matrix}}

First we set \(V^T\) as a counter-clockwise rotation matrix.

\hypertarget{q3b-i-fill-in-the-function-get_rcctheta-to-return-a-2-x-2-matrix-that-when-applied-to-a-vector-x-rotates-it-by-theta-radians-counter-clockwise.}{%
\subsection{Q3b i) Fill in the function ``get\_RCC(theta)'' to return a
2 x 2 matrix that when applied to a vector x rotates it by theta radians
counter
clockwise.}\label{q3b-i-fill-in-the-function-get_rcctheta-to-return-a-2-x-2-matrix-that-when-applied-to-a-vector-x-rotates-it-by-theta-radians-counter-clockwise.}}

Example: If \(V^T = RCC\left(\frac{\pi}{4}\right)\) and x =
\(\begin{bmatrix}1 \\ 0\end{bmatrix}\), then,

\(V^T \begin{bmatrix}1 \\ 0\end{bmatrix}\) =
\(\begin{bmatrix}\frac{1}{\sqrt{2}} \\ \frac{1}{\sqrt{2}} \end{bmatrix}\)

    \begin{Verbatim}[commandchars=\\\{\}]
{\color{incolor}In [{\color{incolor}3}]:} \PY{k}{def} \PY{n+nf}{get\PYZus{}RCC}\PY{p}{(}\PY{n}{theta}\PY{p}{)}\PY{p}{:}
            \PY{l+s+sd}{\PYZsq{}\PYZsq{}\PYZsq{}}
        \PY{l+s+sd}{    Returns a 2 x 2 orthogonal matrix that rotates x by theta radians counter\PYZhy{}clockwise}
        \PY{l+s+sd}{    \PYZsq{}\PYZsq{}\PYZsq{}}
            
            \PY{n}{RCC} \PY{o}{=} \PY{n}{np}\PY{o}{.}\PY{n}{array}\PY{p}{(}\PY{p}{[}\PY{p}{[}\PY{n}{np}\PY{o}{.}\PY{n}{cos}\PY{p}{(}\PY{n}{theta}\PY{p}{)}\PY{p}{,} \PY{o}{\PYZhy{}}\PY{n}{np}\PY{o}{.}\PY{n}{sin}\PY{p}{(}\PY{n}{theta}\PY{p}{)}\PY{p}{]}\PY{p}{,} \PY{p}{[}\PY{n}{np}\PY{o}{.}\PY{n}{sin}\PY{p}{(}\PY{n}{theta}\PY{p}{)}\PY{p}{,} \PY{n}{np}\PY{o}{.}\PY{n}{cos}\PY{p}{(}\PY{n}{theta}\PY{p}{)}\PY{p}{]}\PY{p}{]}\PY{p}{)} \PY{c+c1}{\PYZsh{}TODO: Solution to Q3b i. Change this line by filling in the correct reflection matrix}
        
            \PY{c+c1}{\PYZsh{}\PYZsh{}\PYZsh{}\PYZsh{}\PYZsh{}\PYZsh{}\PYZsh{}\PYZsh{}\PYZsh{}\PYZsh{}\PYZsh{}\PYZsh{}\PYZsh{}\PYZsh{}\PYZsh{}\PYZsh{}\PYZsh{}\PYZsh{}\PYZsh{}\PYZsh{}\PYZsh{}\PYZsh{}\PYZsh{}\PYZsh{}\PYZsh{}\PYZsh{}\PYZsh{}\PYZsh{}\PYZsh{}\PYZsh{}\PYZsh{}\PYZsh{}\PYZsh{}\PYZsh{}\PYZsh{}\PYZsh{}\PYZsh{}\PYZsh{}\PYZsh{}\PYZsh{}\PYZsh{}\PYZsh{}\PYZsh{}\PYZsh{}\PYZsh{}\PYZsh{}\PYZsh{}\PYZsh{}\PYZsh{}\PYZsh{}\PYZsh{}\PYZsh{}\PYZsh{}\PYZsh{}\PYZsh{}\PYZsh{}\PYZsh{}\PYZsh{}\PYZsh{}\PYZsh{}\PYZsh{}\PYZsh{}\PYZsh{}\PYZsh{}\PYZsh{}\PYZsh{}\PYZsh{}\PYZsh{}\PYZsh{}\PYZsh{}\PYZsh{}\PYZsh{}\PYZsh{}\PYZsh{}\PYZsh{}\PYZsh{}\PYZsh{}\PYZsh{}\PYZsh{}\PYZsh{}\PYZsh{}\PYZsh{}\PYZsh{}\PYZsh{}\PYZsh{}\PYZsh{}\PYZsh{}\PYZsh{}\PYZsh{}\PYZsh{}\PYZsh{}\PYZsh{}}
            \PY{c+c1}{\PYZsh{}Some assertions (WARNING: Do not modify below code)}
            \PY{k}{if} \PY{n}{DISABLE\PYZus{}CHECKS} \PY{o+ow}{is} \PY{k+kc}{False}\PY{p}{:}
                \PY{k}{if} \PY{o+ow}{not} \PY{n+nb}{isinstance}\PY{p}{(}\PY{n}{RCC}\PY{p}{,} \PY{n}{np}\PY{o}{.}\PY{n}{ndarray}\PY{p}{)}\PY{p}{:}
                        \PY{k}{raise} \PY{n+ne}{ValueError}\PY{p}{(}\PY{l+s+s1}{\PYZsq{}}\PY{l+s+s1}{RCC must be a np.ndarray}\PY{l+s+s1}{\PYZsq{}}\PY{p}{)}
                \PY{k}{if} \PY{n+nb}{len}\PY{p}{(}\PY{n}{RCC}\PY{o}{.}\PY{n}{shape}\PY{p}{)} \PY{o}{!=} \PY{l+m+mi}{2} \PY{o+ow}{or} \PY{p}{(}\PY{n}{RCC}\PY{o}{.}\PY{n}{shape} \PY{o}{!=} \PY{n}{np}\PY{o}{.}\PY{n}{array}\PY{p}{(}\PY{p}{[}\PY{l+m+mi}{2}\PY{p}{,}\PY{l+m+mi}{2}\PY{p}{]}\PY{p}{)}\PY{p}{)}\PY{o}{.}\PY{n}{any}\PY{p}{(}\PY{p}{)}\PY{p}{:}
                        \PY{k}{raise} \PY{n+ne}{ValueError}\PY{p}{(}\PY{l+s+s1}{\PYZsq{}}\PY{l+s+s1}{RCC must have shape [2,2]}\PY{l+s+s1}{\PYZsq{}}\PY{p}{)}   
            \PY{k}{return} \PY{n}{RCC}
\end{Verbatim}


    \hypertarget{todo-fill-in-solution-to-q3b-i.-in-cell-above}{%
\subsection{\#TODO Fill in solution to Q3b i. in cell
above}\label{todo-fill-in-solution-to-q3b-i.-in-cell-above}}

    \hypertarget{get_rcctheta-function-test}{%
\subsubsection{get\_RCC(theta) function
test}\label{get_rcctheta-function-test}}

If the function get\_RCC(theta) is defined correctly then you should not
get any ERROR statement here.

    \begin{Verbatim}[commandchars=\\\{\}]
{\color{incolor}In [{\color{incolor}4}]:} \PY{n}{x} \PY{o}{=} \PY{n}{np}\PY{o}{.}\PY{n}{array}\PY{p}{(}\PY{p}{[}\PY{p}{[}\PY{l+m+mi}{1}\PY{p}{,}\PY{l+m+mi}{0}\PY{p}{]}\PY{p}{]}\PY{p}{)}\PY{o}{.}\PY{n}{T}
        \PY{n}{V\PYZus{}test} \PY{o}{=} \PY{n}{get\PYZus{}RCC}\PY{p}{(}\PY{n}{np}\PY{o}{.}\PY{n}{pi}\PY{o}{/}\PY{l+m+mi}{4}\PY{p}{)}
        \PY{n}{y} \PY{o}{=} \PY{n}{np}\PY{o}{.}\PY{n}{matmul}\PY{p}{(}\PY{n}{V\PYZus{}test}\PY{p}{,} \PY{n}{x}\PY{p}{)}
        \PY{n}{expected\PYZus{}y} \PY{o}{=} \PY{n}{np}\PY{o}{.}\PY{n}{array}\PY{p}{(}\PY{p}{[}\PY{p}{[}\PY{l+m+mi}{1}\PY{o}{/}\PY{n}{np}\PY{o}{.}\PY{n}{sqrt}\PY{p}{(}\PY{l+m+mi}{2}\PY{p}{)}\PY{p}{,} \PY{l+m+mi}{1}\PY{o}{/}\PY{n}{np}\PY{o}{.}\PY{n}{sqrt}\PY{p}{(}\PY{l+m+mi}{2}\PY{p}{)}\PY{p}{]}\PY{p}{]}\PY{p}{)}\PY{o}{.}\PY{n}{T}
        \PY{n+nb}{print}\PY{p}{(}\PY{l+s+s2}{\PYZdq{}}\PY{l+s+s2}{y:}\PY{l+s+s2}{\PYZdq{}}\PY{p}{)}
        \PY{n+nb}{print}\PY{p}{(}\PY{n}{y}\PY{p}{)}
        \PY{n+nb}{print}\PY{p}{(}\PY{l+s+s2}{\PYZdq{}}\PY{l+s+s2}{Expected y:}\PY{l+s+s2}{\PYZdq{}}\PY{p}{)}
        \PY{n+nb}{print}\PY{p}{(}\PY{n}{expected\PYZus{}y}\PY{p}{)}
        \PY{k}{if} \PY{o+ow}{not} \PY{n}{matrix\PYZus{}equals}\PY{p}{(}\PY{n}{y}\PY{p}{,} \PY{n}{expected\PYZus{}y}\PY{p}{)}\PY{p}{:}
            \PY{n+nb}{print}\PY{p}{(}\PY{l+s+s2}{\PYZdq{}}\PY{l+s+s2}{ERROR: y does not match expected\PYZus{}y. Check if function get\PYZus{}RCC(theta) is completed correctly}\PY{l+s+s2}{\PYZdq{}}\PY{p}{)}
        \PY{k}{else}\PY{p}{:}
            \PY{n+nb}{print}\PY{p}{(}\PY{l+s+s2}{\PYZdq{}}\PY{l+s+s2}{MATCHED: y matches expected\PYZus{}y!}\PY{l+s+s2}{\PYZdq{}}\PY{p}{)}
\end{Verbatim}


    \begin{Verbatim}[commandchars=\\\{\}]
y:
[[0.70710678]
 [0.70710678]]
Expected y:
[[0.70710678]
 [0.70710678]]
MATCHED: y matches expected\_y!

    \end{Verbatim}

    Next we observe how \(V^T\) transforms the unit circle and unit basis
vectors when:

\begin{enumerate}
\def\labelenumi{\arabic{enumi})}
\tightlist
\item
  \(V^T = RCC\left(\frac{\pi}{4}\right)\)
\end{enumerate}

    \begin{Verbatim}[commandchars=\\\{\}]
{\color{incolor}In [{\color{incolor}5}]:} \PY{n}{VT\PYZus{}1} \PY{o}{=} \PY{n}{get\PYZus{}RCC}\PY{p}{(}\PY{n}{np}\PY{o}{.}\PY{n}{pi}\PY{o}{/}\PY{l+m+mi}{4}\PY{p}{)}
        \PY{n}{visualize}\PY{p}{(}\PY{n}{VT} \PY{o}{=} \PY{n}{VT\PYZus{}1}\PY{p}{,} \PY{n}{show\PYZus{}DVT}\PY{o}{=}\PY{k+kc}{False}\PY{p}{,} \PY{n}{show\PYZus{}UDVT}\PY{o}{=}\PY{k+kc}{False}\PY{p}{)}
\end{Verbatim}


    \begin{center}
    \adjustimage{max size={0.9\linewidth}{0.9\paperheight}}{output_11_0.png}
    \end{center}
    { \hspace*{\fill} \\}
    
    \begin{center}
    \adjustimage{max size={0.9\linewidth}{0.9\paperheight}}{output_11_1.png}
    \end{center}
    { \hspace*{\fill} \\}
    
    \begin{enumerate}
\def\labelenumi{\arabic{enumi})}
\setcounter{enumi}{1}
\tightlist
\item
  \(V^T = RCC\left(\frac{-\pi}{3}\right)\)
\end{enumerate}

    \begin{Verbatim}[commandchars=\\\{\}]
{\color{incolor}In [{\color{incolor}6}]:} \PY{n}{VT\PYZus{}2} \PY{o}{=} \PY{n}{get\PYZus{}RCC}\PY{p}{(}\PY{o}{\PYZhy{}}\PY{n}{np}\PY{o}{.}\PY{n}{pi}\PY{o}{/}\PY{l+m+mi}{3}\PY{p}{)}
        \PY{n}{visualize}\PY{p}{(}\PY{n}{VT} \PY{o}{=} \PY{n}{VT\PYZus{}2}\PY{p}{,} \PY{n}{show\PYZus{}DVT}\PY{o}{=}\PY{k+kc}{False}\PY{p}{,} \PY{n}{show\PYZus{}UDVT}\PY{o}{=}\PY{k+kc}{False}\PY{p}{)}
\end{Verbatim}


    \begin{center}
    \adjustimage{max size={0.9\linewidth}{0.9\paperheight}}{output_13_0.png}
    \end{center}
    { \hspace*{\fill} \\}
    
    \begin{center}
    \adjustimage{max size={0.9\linewidth}{0.9\paperheight}}{output_13_1.png}
    \end{center}
    { \hspace*{\fill} \\}
    
    Next we consider the case where \(V^T\) is a reflection matrix.

    \hypertarget{vt-as-a-relfection-matrix}{%
\subsection{\texorpdfstring{\(V^T\) as a relfection
matrix}{V\^{}T as a relfection matrix}}\label{vt-as-a-relfection-matrix}}

A reflection matrix is another type of orthogonal matrix.

\hypertarget{q3b-ii-fill-in-the-function-get_rfx-to-return-a-2-x-2-matrix-that-when-applied-to-a-vector-x-reflects-it-about-the-x-axis.}{%
\subsection{Q3b ii) Fill in the function ``get\_RFx()'' to return a 2 x
2 matrix that when applied to a vector x reflects it about the
x-axis.}\label{q3b-ii-fill-in-the-function-get_rfx-to-return-a-2-x-2-matrix-that-when-applied-to-a-vector-x-reflects-it-about-the-x-axis.}}

Example: If \(V^T =RFx()\) and
\(x = \begin{bmatrix}1 \\ 1\end{bmatrix}\), then,

\(V^T \begin{bmatrix}1 \\ 1\end{bmatrix} = \begin{bmatrix}1 \\ -1\end{bmatrix}\)

    \begin{Verbatim}[commandchars=\\\{\}]
{\color{incolor}In [{\color{incolor}7}]:} \PY{k}{def} \PY{n+nf}{get\PYZus{}RFx}\PY{p}{(}\PY{p}{)}\PY{p}{:}
            \PY{l+s+sd}{\PYZsq{}\PYZsq{}\PYZsq{}}
        \PY{l+s+sd}{    Returns a 2 x 2 orthogonal matrix that reflects about x\PYZhy{}axis}
        \PY{l+s+sd}{    \PYZsq{}\PYZsq{}\PYZsq{}}
        
            
            \PY{n}{RFx} \PY{o}{=} \PY{n}{np}\PY{o}{.}\PY{n}{array}\PY{p}{(}\PY{p}{[}\PY{p}{[}\PY{l+m+mi}{1}\PY{p}{,} \PY{l+m+mi}{0}\PY{p}{]}\PY{p}{,}\PY{p}{[}\PY{l+m+mi}{0}\PY{p}{,} \PY{o}{\PYZhy{}}\PY{l+m+mi}{1}\PY{p}{]}\PY{p}{]}\PY{p}{)}
        
            \PY{c+c1}{\PYZsh{}\PYZsh{}\PYZsh{}\PYZsh{}\PYZsh{}\PYZsh{}\PYZsh{}\PYZsh{}\PYZsh{}\PYZsh{}\PYZsh{}\PYZsh{}\PYZsh{}\PYZsh{}\PYZsh{}\PYZsh{}\PYZsh{}\PYZsh{}\PYZsh{}\PYZsh{}\PYZsh{}\PYZsh{}\PYZsh{}\PYZsh{}\PYZsh{}\PYZsh{}\PYZsh{}\PYZsh{}\PYZsh{}\PYZsh{}\PYZsh{}\PYZsh{}\PYZsh{}\PYZsh{}\PYZsh{}\PYZsh{}\PYZsh{}\PYZsh{}\PYZsh{}\PYZsh{}\PYZsh{}\PYZsh{}\PYZsh{}\PYZsh{}\PYZsh{}\PYZsh{}\PYZsh{}\PYZsh{}\PYZsh{}\PYZsh{}\PYZsh{}\PYZsh{}\PYZsh{}\PYZsh{}\PYZsh{}\PYZsh{}\PYZsh{}\PYZsh{}\PYZsh{}\PYZsh{}\PYZsh{}\PYZsh{}\PYZsh{}\PYZsh{}\PYZsh{}\PYZsh{}\PYZsh{}\PYZsh{}\PYZsh{}\PYZsh{}\PYZsh{}\PYZsh{}\PYZsh{}\PYZsh{}\PYZsh{}\PYZsh{}\PYZsh{}\PYZsh{}\PYZsh{}\PYZsh{}\PYZsh{}\PYZsh{}\PYZsh{}\PYZsh{}\PYZsh{}\PYZsh{}\PYZsh{}\PYZsh{}\PYZsh{}\PYZsh{}\PYZsh{}\PYZsh{}}
            \PY{c+c1}{\PYZsh{}Some assertions (WARNING: Do not modify below code)}
            \PY{k}{if} \PY{n}{DISABLE\PYZus{}CHECKS} \PY{o+ow}{is} \PY{k+kc}{False}\PY{p}{:}
                \PY{k}{if} \PY{o+ow}{not} \PY{n+nb}{isinstance}\PY{p}{(}\PY{n}{RFx}\PY{p}{,} \PY{n}{np}\PY{o}{.}\PY{n}{ndarray}\PY{p}{)}\PY{p}{:}
                        \PY{k}{raise} \PY{n+ne}{ValueError}\PY{p}{(}\PY{l+s+s1}{\PYZsq{}}\PY{l+s+s1}{RFx must be a np.ndarray}\PY{l+s+s1}{\PYZsq{}}\PY{p}{)}
                \PY{k}{if} \PY{n+nb}{len}\PY{p}{(}\PY{n}{RFx}\PY{o}{.}\PY{n}{shape}\PY{p}{)} \PY{o}{!=} \PY{l+m+mi}{2} \PY{o+ow}{or} \PY{p}{(}\PY{n}{RFx}\PY{o}{.}\PY{n}{shape} \PY{o}{!=} \PY{n}{np}\PY{o}{.}\PY{n}{array}\PY{p}{(}\PY{p}{[}\PY{l+m+mi}{2}\PY{p}{,}\PY{l+m+mi}{2}\PY{p}{]}\PY{p}{)}\PY{p}{)}\PY{o}{.}\PY{n}{any}\PY{p}{(}\PY{p}{)}\PY{p}{:}
                        \PY{k}{raise} \PY{n+ne}{ValueError}\PY{p}{(}\PY{l+s+s1}{\PYZsq{}}\PY{l+s+s1}{RFx must have shape [2,2]}\PY{l+s+s1}{\PYZsq{}}\PY{p}{)} 
            \PY{k}{return} \PY{n}{RFx}
\end{Verbatim}


    \hypertarget{todo-fill-in-solution-to-q3b-ii.-in-cell-above}{%
\subsection{\#TODO Fill in solution to Q3b ii. in cell
above}\label{todo-fill-in-solution-to-q3b-ii.-in-cell-above}}

    \hypertarget{get_rfx-function-test}{%
\subsubsection{get\_RFx() function test}\label{get_rfx-function-test}}

If the function get\_RFx() is defined correctly then you should see a
MATCHED statement here.

    \begin{Verbatim}[commandchars=\\\{\}]
{\color{incolor}In [{\color{incolor}8}]:} \PY{n}{x} \PY{o}{=} \PY{n}{np}\PY{o}{.}\PY{n}{array}\PY{p}{(}\PY{p}{[}\PY{p}{[}\PY{l+m+mi}{1}\PY{p}{,}\PY{l+m+mi}{1}\PY{p}{]}\PY{p}{]}\PY{p}{)}\PY{o}{.}\PY{n}{T}
        \PY{n}{V\PYZus{}test} \PY{o}{=} \PY{n}{get\PYZus{}RFx}\PY{p}{(}\PY{p}{)}
        \PY{n}{y} \PY{o}{=} \PY{n}{np}\PY{o}{.}\PY{n}{matmul}\PY{p}{(}\PY{n}{V\PYZus{}test}\PY{p}{,} \PY{n}{x}\PY{p}{)}
        \PY{n}{expected\PYZus{}y} \PY{o}{=} \PY{n}{np}\PY{o}{.}\PY{n}{array}\PY{p}{(}\PY{p}{[}\PY{p}{[}\PY{l+m+mi}{1}\PY{p}{,} \PY{o}{\PYZhy{}}\PY{l+m+mi}{1}\PY{p}{]}\PY{p}{]}\PY{p}{)}\PY{o}{.}\PY{n}{T}
        \PY{n+nb}{print}\PY{p}{(}\PY{l+s+s2}{\PYZdq{}}\PY{l+s+s2}{y:}\PY{l+s+s2}{\PYZdq{}}\PY{p}{)}
        \PY{n+nb}{print}\PY{p}{(}\PY{n}{y}\PY{p}{)}
        \PY{n+nb}{print}\PY{p}{(}\PY{l+s+s2}{\PYZdq{}}\PY{l+s+s2}{Expected y:}\PY{l+s+s2}{\PYZdq{}}\PY{p}{)}
        \PY{n+nb}{print}\PY{p}{(}\PY{n}{expected\PYZus{}y}\PY{p}{)}
        \PY{k}{if} \PY{o+ow}{not} \PY{n}{matrix\PYZus{}equals}\PY{p}{(}\PY{n}{y}\PY{p}{,} \PY{n}{expected\PYZus{}y}\PY{p}{)}\PY{p}{:}
            \PY{n+nb}{print}\PY{p}{(}\PY{l+s+s2}{\PYZdq{}}\PY{l+s+s2}{ERROR: y does not match expected\PYZus{}y. Check if function get\PYZus{}RFx() is completed correctly}\PY{l+s+s2}{\PYZdq{}}\PY{p}{)}
        \PY{k}{else}\PY{p}{:}
            \PY{n+nb}{print}\PY{p}{(}\PY{l+s+s2}{\PYZdq{}}\PY{l+s+s2}{MATCHED: y matches expected\PYZus{}y!}\PY{l+s+s2}{\PYZdq{}}\PY{p}{)}
\end{Verbatim}


    \begin{Verbatim}[commandchars=\\\{\}]
y:
[[ 1]
 [-1]]
Expected y:
[[ 1]
 [-1]]
MATCHED: y matches expected\_y!

    \end{Verbatim}

    \(V^T = RFx()\)

    \begin{Verbatim}[commandchars=\\\{\}]
{\color{incolor}In [{\color{incolor}9}]:} \PY{n}{VT\PYZus{}3} \PY{o}{=} \PY{n}{get\PYZus{}RFx}\PY{p}{(}\PY{p}{)}
        \PY{n}{visualize}\PY{p}{(}\PY{n}{VT} \PY{o}{=} \PY{n}{VT\PYZus{}3}\PY{p}{,} \PY{n}{show\PYZus{}DVT}\PY{o}{=}\PY{k+kc}{False}\PY{p}{,} \PY{n}{show\PYZus{}UDVT}\PY{o}{=}\PY{k+kc}{False}\PY{p}{)}
\end{Verbatim}


    \begin{center}
    \adjustimage{max size={0.9\linewidth}{0.9\paperheight}}{output_21_0.png}
    \end{center}
    { \hspace*{\fill} \\}
    
    \begin{center}
    \adjustimage{max size={0.9\linewidth}{0.9\paperheight}}{output_21_1.png}
    \end{center}
    { \hspace*{\fill} \\}
    
    \hypertarget{vt-as-a-composition-of-reflection-and-rotation-matrix}{%
\subsection{\texorpdfstring{\(V^T\) as a composition of reflection and
rotation
matrix}{V\^{}T as a composition of reflection and rotation matrix}}\label{vt-as-a-composition-of-reflection-and-rotation-matrix}}

In general an orthogonal transformation can be viewed as compositions of
rotation and reflection operators Next we observe the effect of setting

\(V^T = RFx()RCC\left(\frac{\pi}{4}\right)\)

    \begin{Verbatim}[commandchars=\\\{\}]
{\color{incolor}In [{\color{incolor}10}]:} \PY{n}{VT\PYZus{}4} \PY{o}{=} \PY{n}{np}\PY{o}{.}\PY{n}{matmul}\PY{p}{(}\PY{n}{VT\PYZus{}3}\PY{p}{,} \PY{n}{VT\PYZus{}1}\PY{p}{)}
         \PY{c+c1}{\PYZsh{}Check that VT\PYZus{}4 is still orthogonal}
         \PY{n+nb}{print}\PY{p}{(}\PY{l+s+s2}{\PYZdq{}}\PY{l+s+s2}{VT\PYZus{}4 is orthogonal?: }\PY{l+s+s2}{\PYZdq{}}\PY{p}{,} \PY{n}{is\PYZus{}orthogonal}\PY{p}{(}\PY{n}{VT\PYZus{}4}\PY{p}{)}\PY{p}{)}
         \PY{n}{visualize}\PY{p}{(}\PY{n}{VT} \PY{o}{=} \PY{n}{VT\PYZus{}4}\PY{p}{,} \PY{n}{show\PYZus{}DVT}\PY{o}{=}\PY{k+kc}{False}\PY{p}{,} \PY{n}{show\PYZus{}UDVT}\PY{o}{=}\PY{k+kc}{False}\PY{p}{)}
\end{Verbatim}


    \begin{Verbatim}[commandchars=\\\{\}]
VT\_4 is orthogonal?:  True

    \end{Verbatim}

    \begin{center}
    \adjustimage{max size={0.9\linewidth}{0.9\paperheight}}{output_23_1.png}
    \end{center}
    { \hspace*{\fill} \\}
    
    \begin{center}
    \adjustimage{max size={0.9\linewidth}{0.9\paperheight}}{output_23_2.png}
    \end{center}
    { \hspace*{\fill} \\}
    
    \hypertarget{q3b-iii-comment-on-the-effect-of-vt-rccleftfracpi4rightrfx.-is-it-same-as-the-case-when-vt-rfxrccleftfracpi4right}{%
\subsection{\texorpdfstring{Q3b iii) Comment on the effect of
\(V^T = RCC\left(\frac{\pi}{4}\right)RFx()\). Is it same as the case
when
\(V^T = RFx()RCC\left(\frac{\pi}{4}\right)\)?}{Q3b iii) Comment on the effect of V\^{}T = RCC\textbackslash{}left(\textbackslash{}frac\{\textbackslash{}pi\}\{4\}\textbackslash{}right)RFx(). Is it same as the case when V\^{}T = RFx()RCC\textbackslash{}left(\textbackslash{}frac\{\textbackslash{}pi\}\{4\}\textbackslash{}right)?}}\label{q3b-iii-comment-on-the-effect-of-vt-rccleftfracpi4rightrfx.-is-it-same-as-the-case-when-vt-rfxrccleftfracpi4right}}

    \begin{Verbatim}[commandchars=\\\{\}]
{\color{incolor}In [{\color{incolor}11}]:} \PY{n}{VT\PYZus{}5} \PY{o}{=}  \PY{n}{np}\PY{o}{.}\PY{n}{matmul}\PY{p}{(}\PY{n}{VT\PYZus{}1}\PY{p}{,} \PY{n}{VT\PYZus{}3}\PY{p}{)}
         \PY{n}{visualize}\PY{p}{(}\PY{n}{VT} \PY{o}{=} \PY{n}{VT\PYZus{}5}\PY{p}{,} \PY{n}{show\PYZus{}DVT}\PY{o}{=}\PY{k+kc}{False}\PY{p}{,} \PY{n}{show\PYZus{}UDVT}\PY{o}{=}\PY{k+kc}{False}\PY{p}{)}
\end{Verbatim}


    \begin{center}
    \adjustimage{max size={0.9\linewidth}{0.9\paperheight}}{output_25_0.png}
    \end{center}
    { \hspace*{\fill} \\}
    
    \begin{center}
    \adjustimage{max size={0.9\linewidth}{0.9\paperheight}}{output_25_1.png}
    \end{center}
    { \hspace*{\fill} \\}
    
    \hypertarget{todo-fill-in-solution-to-q3b-iii-here}{%
\subsection{\#TODO: Fill in solution to Q3b iii
here}\label{todo-fill-in-solution-to-q3b-iii-here}}

    It is not the same because the order of matrix multiplication matters.
It is not commutative.

    \hypertarget{effect-of-linear-transformation-by-diagonal-matrix-d}{%
\section{Effect of linear transformation by diagonal matrix
D}\label{effect-of-linear-transformation-by-diagonal-matrix-d}}

The diagonal matrix D with entries \(\sigma_1\) and \(\sigma_2\),
transforms the unit circle into an ellipse with x direction scaled by
\(\sigma_1\) and y direction scaled by \(\sigma_2\).

If \(\sigma_1 > \sigma_2\), then the major axis of the ellipse will be
along the x-axis.

If \(\sigma_1 < \sigma_2\), then the major axis of the ellipse will be
along the y-axis.

If \(\sigma_1 = \sigma_2\), then the ellipse will have both axis equal
(i.e it is a circle).

Note that multiplying by D, does not rotate or reflect points in any
way. It is a purely scaling operation where different directions get
scaled by different values based on entries of D.

    \hypertarget{q3c-i-comment-on-the-length-of-major-and-minor-axis-of-the-ellipse-and-their-orientation-with-respect-to-x-and-y-axis-when-d-has-entries-3-2.}{%
\subsection{Q3c i: Comment on the length of major and minor axis of the
ellipse and their orientation with respect to X and Y axis when D has
entries {[}3,
2{]}.}\label{q3c-i-comment-on-the-length-of-major-and-minor-axis-of-the-ellipse-and-their-orientation-with-respect-to-x-and-y-axis-when-d-has-entries-3-2.}}

    \begin{Verbatim}[commandchars=\\\{\}]
{\color{incolor}In [{\color{incolor}12}]:} \PY{n}{D\PYZus{}1} \PY{o}{=} \PY{n}{np}\PY{o}{.}\PY{n}{array}\PY{p}{(}\PY{p}{[}\PY{l+m+mi}{3}\PY{p}{,} \PY{l+m+mi}{2}\PY{p}{]}\PY{p}{)}
         \PY{n}{visualize}\PY{p}{(} \PY{n}{D} \PY{o}{=} \PY{n}{D\PYZus{}1}\PY{p}{,} \PY{n}{show\PYZus{}original}\PY{o}{=}\PY{k+kc}{False}\PY{p}{,} \PY{n}{show\PYZus{}UDVT}\PY{o}{=}\PY{k+kc}{False}\PY{p}{)}
\end{Verbatim}


    \begin{center}
    \adjustimage{max size={0.9\linewidth}{0.9\paperheight}}{output_30_0.png}
    \end{center}
    { \hspace*{\fill} \\}
    
    \begin{center}
    \adjustimage{max size={0.9\linewidth}{0.9\paperheight}}{output_30_1.png}
    \end{center}
    { \hspace*{\fill} \\}
    
    \hypertarget{todo-solution-to-q3c-i}{%
\subsection{\#TODO: Solution to Q3c i}\label{todo-solution-to-q3c-i}}

    The major axis is x-axis while the minor is y-axis. It can be easily
guessed by the entries of D. \(\sigma_1 = 3\) and \(\sigma_2 = 2\). The
minor-axis is stretched twice while major-axis did by thrice.

    \hypertarget{q3c-ii-comment-on-the-length-of-major-and-minor-axis-of-the-ellipse-and-their-orientation-with-respect-to-x-and-y-axis-when-d-has-entries-2-3.}{%
\subsection{Q3c ii: Comment on the length of major and minor axis of the
ellipse and their orientation with respect to X and Y axis when D has
entries {[}2,
3{]}.}\label{q3c-ii-comment-on-the-length-of-major-and-minor-axis-of-the-ellipse-and-their-orientation-with-respect-to-x-and-y-axis-when-d-has-entries-2-3.}}

    \begin{Verbatim}[commandchars=\\\{\}]
{\color{incolor}In [{\color{incolor}13}]:} \PY{n}{D\PYZus{}2} \PY{o}{=} \PY{n}{np}\PY{o}{.}\PY{n}{array}\PY{p}{(}\PY{p}{[}\PY{l+m+mi}{2}\PY{p}{,} \PY{l+m+mi}{3}\PY{p}{]}\PY{p}{)}
         \PY{n}{visualize}\PY{p}{(} \PY{n}{D} \PY{o}{=} \PY{n}{D\PYZus{}2}\PY{p}{,} \PY{n}{VT} \PY{o}{=} \PY{n}{get\PYZus{}RCC}\PY{p}{(}\PY{n}{np}\PY{o}{.}\PY{n}{pi}\PY{o}{/}\PY{l+m+mi}{4}\PY{p}{)}\PY{p}{,} \PY{n}{show\PYZus{}original}\PY{o}{=}\PY{k+kc}{False}\PY{p}{,} \PY{n}{show\PYZus{}UDVT}\PY{o}{=}\PY{k+kc}{False}\PY{p}{)}
\end{Verbatim}


    \begin{center}
    \adjustimage{max size={0.9\linewidth}{0.9\paperheight}}{output_34_0.png}
    \end{center}
    { \hspace*{\fill} \\}
    
    \begin{center}
    \adjustimage{max size={0.9\linewidth}{0.9\paperheight}}{output_34_1.png}
    \end{center}
    { \hspace*{\fill} \\}
    
    \hypertarget{todo-enter-solution-to-q3c-ii-here}{%
\subsection{\#TODO Enter solution to Q3c ii
here}\label{todo-enter-solution-to-q3c-ii-here}}

    The major axis is y-axis and the minor is x-axis by the same reason
above. X-axis got stretched twice while y-axis did by thirce.

    \hypertarget{q3c-iii-what-can-you-say-about-the-ellipse-when-d-has-entries-2-2}{%
\subsection{Q3c iii: What can you say about the ellipse when D has
entries {[}2,
2{]}?}\label{q3c-iii-what-can-you-say-about-the-ellipse-when-d-has-entries-2-2}}

    \begin{Verbatim}[commandchars=\\\{\}]
{\color{incolor}In [{\color{incolor}14}]:} \PY{n}{D\PYZus{}3} \PY{o}{=} \PY{n}{np}\PY{o}{.}\PY{n}{array}\PY{p}{(}\PY{p}{[}\PY{l+m+mi}{2}\PY{p}{,} \PY{l+m+mi}{2}\PY{p}{]}\PY{p}{)}
         \PY{n}{visualize}\PY{p}{(} \PY{n}{D} \PY{o}{=} \PY{n}{D\PYZus{}3}\PY{p}{,} \PY{n}{VT} \PY{o}{=} \PY{n}{get\PYZus{}RCC}\PY{p}{(}\PY{o}{\PYZhy{}}\PY{n}{np}\PY{o}{.}\PY{n}{pi}\PY{o}{/}\PY{l+m+mi}{3}\PY{p}{)}\PY{p}{,} \PY{n}{show\PYZus{}original}\PY{o}{=}\PY{k+kc}{False}\PY{p}{,} \PY{n}{show\PYZus{}UDVT}\PY{o}{=}\PY{k+kc}{False}\PY{p}{)}
\end{Verbatim}


    \begin{center}
    \adjustimage{max size={0.9\linewidth}{0.9\paperheight}}{output_38_0.png}
    \end{center}
    { \hspace*{\fill} \\}
    
    \begin{center}
    \adjustimage{max size={0.9\linewidth}{0.9\paperheight}}{output_38_1.png}
    \end{center}
    { \hspace*{\fill} \\}
    
    \hypertarget{todo-enter-solution-to-q3c-iii-here}{%
\subsection{\#TODO Enter solution to Q3c iii
here}\label{todo-enter-solution-to-q3c-iii-here}}

    It's a circle, yay!. which means that their axis are equal and there are
no major nor minor axis.

    \hypertarget{q3c-iv-what-can-you-say-about-the-ellipse-when-d-has-entries-2-0}{%
\subsection{Q3c iv: What can you say about the ellipse when D has
entries {[}2,
0{]}?}\label{q3c-iv-what-can-you-say-about-the-ellipse-when-d-has-entries-2-0}}

    \begin{Verbatim}[commandchars=\\\{\}]
{\color{incolor}In [{\color{incolor}15}]:} \PY{n}{D\PYZus{}4} \PY{o}{=} \PY{n}{np}\PY{o}{.}\PY{n}{array}\PY{p}{(}\PY{p}{[}\PY{l+m+mi}{2}\PY{p}{,} \PY{l+m+mi}{0}\PY{p}{]}\PY{p}{)}
         \PY{n}{visualize}\PY{p}{(} \PY{n}{D} \PY{o}{=} \PY{n}{D\PYZus{}4}\PY{p}{,} \PY{n}{VT} \PY{o}{=} \PY{n}{get\PYZus{}RCC}\PY{p}{(}\PY{o}{\PYZhy{}}\PY{n}{np}\PY{o}{.}\PY{n}{pi}\PY{o}{/}\PY{l+m+mi}{3}\PY{p}{)}\PY{p}{,} \PY{n}{show\PYZus{}original}\PY{o}{=}\PY{k+kc}{False}\PY{p}{,} \PY{n}{show\PYZus{}UDVT}\PY{o}{=}\PY{k+kc}{False}\PY{p}{)}
\end{Verbatim}


    \begin{center}
    \adjustimage{max size={0.9\linewidth}{0.9\paperheight}}{output_42_0.png}
    \end{center}
    { \hspace*{\fill} \\}
    
    \begin{center}
    \adjustimage{max size={0.9\linewidth}{0.9\paperheight}}{output_42_1.png}
    \end{center}
    { \hspace*{\fill} \\}
    
    \hypertarget{todo-enter-solution-to-q3c-iv-here}{%
\subsection{\#TODO: Enter solution to Q3c iv
here}\label{todo-enter-solution-to-q3c-iv-here}}

    The major axis is x-axis. By multiplying by \(\sigma_2 = 0\), it
flattens the circle and transforms it into a line.

    \hypertarget{effect-of-the-linear-transformation-by-orthogonal-matrix-u}{%
\section{\texorpdfstring{Effect of the linear transformation by
orthogonal matrix
\(U\)}{Effect of the linear transformation by orthogonal matrix U}}\label{effect-of-the-linear-transformation-by-orthogonal-matrix-u}}

As we saw before for \(V^T\), a 2 x 2 orthogonal matrix can be viewed as
a linear transformation that performs some combination of rotations and
reflections.

\hypertarget{q3d-i-comment-on-the-effect-of-u-rccleftfracpi4right-as-in-cell-below.-what-happened-to-the-ellipse-did-length-of-major-and-minor-axis-change}{%
\subsection{\texorpdfstring{Q3d i Comment on the effect of
\(U = RCC\left(\frac{\pi}{4}\right)\) as in cell below. What happened to
the ellipse? Did length of major and minor axis
change?}{Q3d i Comment on the effect of U = RCC\textbackslash{}left(\textbackslash{}frac\{\textbackslash{}pi\}\{4\}\textbackslash{}right) as in cell below. What happened to the ellipse? Did length of major and minor axis change?}}\label{q3d-i-comment-on-the-effect-of-u-rccleftfracpi4right-as-in-cell-below.-what-happened-to-the-ellipse-did-length-of-major-and-minor-axis-change}}

    \begin{Verbatim}[commandchars=\\\{\}]
{\color{incolor}In [{\color{incolor}16}]:} \PY{n}{U\PYZus{}1} \PY{o}{=} \PY{n}{get\PYZus{}RCC}\PY{p}{(}\PY{n}{np}\PY{o}{.}\PY{n}{pi}\PY{o}{/}\PY{l+m+mi}{4}\PY{p}{)}
         \PY{n}{visualize}\PY{p}{(} \PY{n}{U} \PY{o}{=} \PY{n}{U\PYZus{}1}\PY{p}{,} \PY{n}{D} \PY{o}{=}\PY{n}{np}\PY{o}{.}\PY{n}{array}\PY{p}{(}\PY{p}{[}\PY{l+m+mi}{2}\PY{p}{,}\PY{l+m+mi}{3}\PY{p}{]}\PY{p}{)}\PY{p}{,} \PY{n}{show\PYZus{}original}\PY{o}{=}\PY{k+kc}{False}\PY{p}{,} \PY{n}{show\PYZus{}VT}\PY{o}{=}\PY{k+kc}{False}\PY{p}{)}
\end{Verbatim}


    \begin{center}
    \adjustimage{max size={0.9\linewidth}{0.9\paperheight}}{output_46_0.png}
    \end{center}
    { \hspace*{\fill} \\}
    
    \begin{center}
    \adjustimage{max size={0.9\linewidth}{0.9\paperheight}}{output_46_1.png}
    \end{center}
    { \hspace*{\fill} \\}
    
    \hypertarget{todo-enter-solution-to-q3di-here}{%
\subsection{\#TODO: Enter solution to Q3di
here}\label{todo-enter-solution-to-q3di-here}}

    It did not change the major and minor axis. It only rotated the circle
counter-clockwise.

    \hypertarget{q3d-ii-comment-on-the-effect-of-u-rfx-as-in-cell-below.-what-happened-to-the-ellipse-did-length-of-major-and-minor-axis-change}{%
\subsection{\texorpdfstring{Q3d ii Comment on the effect of
\(U = RFx()\) as in cell below. What happened to the ellipse? Did length
of major and minor axis
change?}{Q3d ii Comment on the effect of U = RFx() as in cell below. What happened to the ellipse? Did length of major and minor axis change?}}\label{q3d-ii-comment-on-the-effect-of-u-rfx-as-in-cell-below.-what-happened-to-the-ellipse-did-length-of-major-and-minor-axis-change}}

    \begin{Verbatim}[commandchars=\\\{\}]
{\color{incolor}In [{\color{incolor}17}]:} \PY{n}{U\PYZus{}2} \PY{o}{=} \PY{n}{get\PYZus{}RFx}\PY{p}{(}\PY{p}{)}
         \PY{n}{visualize}\PY{p}{(} \PY{n}{U} \PY{o}{=} \PY{n}{U\PYZus{}2}\PY{p}{,} \PY{n}{D} \PY{o}{=}\PY{n}{np}\PY{o}{.}\PY{n}{array}\PY{p}{(}\PY{p}{[}\PY{l+m+mi}{2}\PY{p}{,}\PY{l+m+mi}{3}\PY{p}{]}\PY{p}{)}\PY{p}{,} \PY{n}{VT} \PY{o}{=} \PY{n}{get\PYZus{}RCC}\PY{p}{(}\PY{n}{np}\PY{o}{.}\PY{n}{pi}\PY{o}{/}\PY{l+m+mi}{4}\PY{p}{)}\PY{p}{,} \PY{n}{show\PYZus{}original}\PY{o}{=}\PY{k+kc}{False}\PY{p}{,} \PY{n}{show\PYZus{}VT}\PY{o}{=}\PY{k+kc}{False}\PY{p}{)}
\end{Verbatim}


    \begin{center}
    \adjustimage{max size={0.9\linewidth}{0.9\paperheight}}{output_50_0.png}
    \end{center}
    { \hspace*{\fill} \\}
    
    \begin{center}
    \adjustimage{max size={0.9\linewidth}{0.9\paperheight}}{output_50_1.png}
    \end{center}
    { \hspace*{\fill} \\}
    
    \hypertarget{todo-fill-in-solution-to-q3d-ii-here}{%
\subsection{\#TODO Fill in solution to Q3d ii
here}\label{todo-fill-in-solution-to-q3d-ii-here}}

    It did not change the length of major and minor axis.

    \hypertarget{putting-everything-together.-effect-of-linear-transformation-by-udvt}{%
\section{\texorpdfstring{Putting everything together. Effect of linear
transformation by
\(UDV^T\)}{Putting everything together. Effect of linear transformation by UDV\^{}T}}\label{putting-everything-together.-effect-of-linear-transformation-by-udvt}}

    \hypertarget{case-i}{%
\subsubsection{Case I}\label{case-i}}

    \begin{Verbatim}[commandchars=\\\{\}]
{\color{incolor}In [{\color{incolor}18}]:} \PY{n}{U} \PY{o}{=} \PY{n}{get\PYZus{}RCC}\PY{p}{(}\PY{n}{np}\PY{o}{.}\PY{n}{pi}\PY{o}{/}\PY{l+m+mi}{4}\PY{p}{)}
         \PY{n}{VT} \PY{o}{=} \PY{n}{get\PYZus{}RCC}\PY{p}{(}\PY{o}{\PYZhy{}}\PY{n}{np}\PY{o}{.}\PY{n}{pi}\PY{o}{/}\PY{l+m+mi}{3}\PY{p}{)}
         \PY{n}{D} \PY{o}{=} \PY{n}{np}\PY{o}{.}\PY{n}{array}\PY{p}{(}\PY{p}{[}\PY{l+m+mi}{3}\PY{p}{,}\PY{l+m+mi}{2}\PY{p}{]}\PY{p}{)}
         \PY{n}{visualize}\PY{p}{(}\PY{n}{U} \PY{o}{=} \PY{n}{U}\PY{p}{,} \PY{n}{VT}\PY{o}{=} \PY{n}{VT}\PY{p}{,} \PY{n}{D}\PY{o}{=}\PY{n}{D}\PY{p}{)}
\end{Verbatim}


    \begin{center}
    \adjustimage{max size={0.9\linewidth}{0.9\paperheight}}{output_55_0.png}
    \end{center}
    { \hspace*{\fill} \\}
    
    \begin{center}
    \adjustimage{max size={0.9\linewidth}{0.9\paperheight}}{output_55_1.png}
    \end{center}
    { \hspace*{\fill} \\}
    
    \begin{center}
    \adjustimage{max size={0.9\linewidth}{0.9\paperheight}}{output_55_2.png}
    \end{center}
    { \hspace*{\fill} \\}
    
    \begin{center}
    \adjustimage{max size={0.9\linewidth}{0.9\paperheight}}{output_55_3.png}
    \end{center}
    { \hspace*{\fill} \\}
    
    The above figures show the transformation after each step.

    \hypertarget{case-ii}{%
\subsubsection{Case II}\label{case-ii}}

    \begin{Verbatim}[commandchars=\\\{\}]
{\color{incolor}In [{\color{incolor}19}]:} \PY{n}{U} \PY{o}{=} \PY{n}{get\PYZus{}RCC}\PY{p}{(}\PY{n}{np}\PY{o}{.}\PY{n}{pi}\PY{o}{/}\PY{l+m+mi}{4}\PY{p}{)}
         \PY{n}{VT} \PY{o}{=} \PY{n}{get\PYZus{}RCC}\PY{p}{(}\PY{o}{\PYZhy{}}\PY{n}{np}\PY{o}{.}\PY{n}{pi}\PY{o}{/}\PY{l+m+mi}{3}\PY{p}{)}
         \PY{n}{D} \PY{o}{=} \PY{n}{np}\PY{o}{.}\PY{n}{array}\PY{p}{(}\PY{p}{[}\PY{l+m+mi}{3}\PY{p}{,}\PY{l+m+mi}{0}\PY{p}{]}\PY{p}{)}
         \PY{n}{visualize}\PY{p}{(}\PY{n}{U} \PY{o}{=} \PY{n}{U}\PY{p}{,} \PY{n}{VT}\PY{o}{=} \PY{n}{VT}\PY{p}{,} \PY{n}{D}\PY{o}{=}\PY{n}{D}\PY{p}{)}
\end{Verbatim}


    \begin{center}
    \adjustimage{max size={0.9\linewidth}{0.9\paperheight}}{output_58_0.png}
    \end{center}
    { \hspace*{\fill} \\}
    
    \begin{center}
    \adjustimage{max size={0.9\linewidth}{0.9\paperheight}}{output_58_1.png}
    \end{center}
    { \hspace*{\fill} \\}
    
    \begin{center}
    \adjustimage{max size={0.9\linewidth}{0.9\paperheight}}{output_58_2.png}
    \end{center}
    { \hspace*{\fill} \\}
    
    \begin{center}
    \adjustimage{max size={0.9\linewidth}{0.9\paperheight}}{output_58_3.png}
    \end{center}
    { \hspace*{\fill} \\}
    
    The above figures show the transformation after each step.

    \hypertarget{case-iii}{%
\subsubsection{Case III}\label{case-iii}}

    \begin{Verbatim}[commandchars=\\\{\}]
{\color{incolor}In [{\color{incolor}20}]:} \PY{n}{A} \PY{o}{=} \PY{n}{np}\PY{o}{.}\PY{n}{array}\PY{p}{(}\PY{p}{[}\PY{p}{[}\PY{l+m+mi}{5}\PY{p}{,} \PY{l+m+mi}{3}\PY{p}{]}\PY{p}{,} \PY{p}{[}\PY{l+m+mi}{2}\PY{p}{,} \PY{o}{\PYZhy{}}\PY{l+m+mi}{2}\PY{p}{]}\PY{p}{]}\PY{p}{)}
         \PY{n}{U}\PY{p}{,}\PY{n}{D}\PY{p}{,}\PY{n}{VT} \PY{o}{=} \PY{n}{np}\PY{o}{.}\PY{n}{linalg}\PY{o}{.}\PY{n}{svd}\PY{p}{(}\PY{n}{A}\PY{p}{)}
         \PY{n}{visualize}\PY{p}{(}\PY{n}{U} \PY{o}{=} \PY{n}{U}\PY{p}{,} \PY{n}{D}\PY{o}{=}\PY{n}{D}\PY{p}{,} \PY{n}{VT}\PY{o}{=}\PY{n}{VT}\PY{p}{)}
\end{Verbatim}


    \begin{center}
    \adjustimage{max size={0.9\linewidth}{0.9\paperheight}}{output_61_0.png}
    \end{center}
    { \hspace*{\fill} \\}
    
    \begin{center}
    \adjustimage{max size={0.9\linewidth}{0.9\paperheight}}{output_61_1.png}
    \end{center}
    { \hspace*{\fill} \\}
    
    \begin{center}
    \adjustimage{max size={0.9\linewidth}{0.9\paperheight}}{output_61_2.png}
    \end{center}
    { \hspace*{\fill} \\}
    
    \begin{center}
    \adjustimage{max size={0.9\linewidth}{0.9\paperheight}}{output_61_3.png}
    \end{center}
    { \hspace*{\fill} \\}
    
    \hypertarget{q3e-for-case-iii-based-on-the-figures-obtained-by-running-the-cell-answer-the-following-questions}{%
\subsection{Q3e For case III, based on the figures obtained by running
the cell, answer the following
questions:}\label{q3e-for-case-iii-based-on-the-figures-obtained-by-running-the-cell-answer-the-following-questions}}

\begin{enumerate}
\def\labelenumi{\arabic{enumi})}
\item
  Is \(V^T\) a pure rotation, pure reflection or combination of both?
\item
  Let \(\sigma_1\) and \(\sigma_2\) denote the entries of the diagonal
  matrix in SVD of A, with \(\sigma_1 > \sigma_2\)? What is an
  approximate value of \(\frac{\sigma_1}{\sigma_2}\)?
\item
  Is \(U\) a pure rotation, pure reflection or combination of both?
\end{enumerate}

    \hypertarget{todo-enter-solution-to-q3e-here}{%
\subsection{\#TODO Enter solution to Q3e
here}\label{todo-enter-solution-to-q3e-here}}

    \begin{Verbatim}[commandchars=\\\{\}]
{\color{incolor}In [{\color{incolor}21}]:} \PY{c+c1}{\PYZsh{} Ratio of entries}
         \PY{n}{D}\PY{p}{[}\PY{l+m+mi}{0}\PY{p}{]} \PY{o}{/} \PY{n}{D}\PY{p}{[}\PY{l+m+mi}{1}\PY{p}{]}
\end{Verbatim}


\begin{Verbatim}[commandchars=\\\{\}]
{\color{outcolor}Out[{\color{outcolor}21}]:} 2.1625919067959654
\end{Verbatim}
            
    \begin{enumerate}
\def\labelenumi{\arabic{enumi}.}
\tightlist
\item
  It is a pure rotation.
\item
  \(\sigma_1 > \sigma_2\) which is why major axis is x-axis. The approx.
  ratio is about 2.16.
\item
  It is a combination. It reflected and rotated counterclockwise a bit.
\end{enumerate}

    \hypertarget{exploration-area-not-part-of-homework-question}{%
\section{Exploration Area (Not part of homework
question)}\label{exploration-area-not-part-of-homework-question}}

You are free to visualize the effect of the SVD transformation on the
unit circle for whatever matrix you desire

    \begin{Verbatim}[commandchars=\\\{\}]
{\color{incolor}In [{\color{incolor}22}]:} \PY{c+c1}{\PYZsh{}Sample format 1}
         \PY{n}{U} \PY{o}{=} \PY{n}{get\PYZus{}RCC}\PY{p}{(}\PY{n}{np}\PY{o}{.}\PY{n}{pi}\PY{o}{/}\PY{l+m+mi}{4}\PY{p}{)}
         \PY{n}{VT} \PY{o}{=} \PY{n}{get\PYZus{}RCC}\PY{p}{(}\PY{o}{\PYZhy{}}\PY{n}{np}\PY{o}{.}\PY{n}{pi}\PY{o}{/}\PY{l+m+mi}{3}\PY{p}{)}
         \PY{n}{D} \PY{o}{=} \PY{n}{np}\PY{o}{.}\PY{n}{array}\PY{p}{(}\PY{p}{[}\PY{l+m+mi}{3}\PY{p}{,}\PY{l+m+mi}{2}\PY{p}{]}\PY{p}{)}
         \PY{n}{visualize}\PY{p}{(}\PY{n}{U} \PY{o}{=} \PY{n}{U}\PY{p}{,} \PY{n}{VT}\PY{o}{=} \PY{n}{VT}\PY{p}{,} \PY{n}{D}\PY{o}{=}\PY{n}{D}\PY{p}{)}
\end{Verbatim}


    \begin{center}
    \adjustimage{max size={0.9\linewidth}{0.9\paperheight}}{output_67_0.png}
    \end{center}
    { \hspace*{\fill} \\}
    
    \begin{center}
    \adjustimage{max size={0.9\linewidth}{0.9\paperheight}}{output_67_1.png}
    \end{center}
    { \hspace*{\fill} \\}
    
    \begin{center}
    \adjustimage{max size={0.9\linewidth}{0.9\paperheight}}{output_67_2.png}
    \end{center}
    { \hspace*{\fill} \\}
    
    \begin{center}
    \adjustimage{max size={0.9\linewidth}{0.9\paperheight}}{output_67_3.png}
    \end{center}
    { \hspace*{\fill} \\}
    
    \begin{Verbatim}[commandchars=\\\{\}]
{\color{incolor}In [{\color{incolor}23}]:} \PY{c+c1}{\PYZsh{}Sample format 2}
         \PY{n}{A} \PY{o}{=} \PY{n}{np}\PY{o}{.}\PY{n}{array}\PY{p}{(}\PY{p}{[}\PY{p}{[}\PY{l+m+mi}{5}\PY{p}{,} \PY{l+m+mi}{3}\PY{p}{]}\PY{p}{,} \PY{p}{[}\PY{l+m+mi}{2}\PY{p}{,} \PY{o}{\PYZhy{}}\PY{l+m+mi}{2}\PY{p}{]}\PY{p}{]}\PY{p}{)}
         \PY{n}{U}\PY{p}{,}\PY{n}{D}\PY{p}{,}\PY{n}{VT} \PY{o}{=} \PY{n}{np}\PY{o}{.}\PY{n}{linalg}\PY{o}{.}\PY{n}{svd}\PY{p}{(}\PY{n}{A}\PY{p}{)}
         \PY{n}{visualize}\PY{p}{(}\PY{n}{U} \PY{o}{=} \PY{n}{U}\PY{p}{,} \PY{n}{D}\PY{o}{=}\PY{n}{D}\PY{p}{,} \PY{n}{VT}\PY{o}{=}\PY{n}{VT}\PY{p}{)}
\end{Verbatim}


    \begin{center}
    \adjustimage{max size={0.9\linewidth}{0.9\paperheight}}{output_68_0.png}
    \end{center}
    { \hspace*{\fill} \\}
    
    \begin{center}
    \adjustimage{max size={0.9\linewidth}{0.9\paperheight}}{output_68_1.png}
    \end{center}
    { \hspace*{\fill} \\}
    
    \begin{center}
    \adjustimage{max size={0.9\linewidth}{0.9\paperheight}}{output_68_2.png}
    \end{center}
    { \hspace*{\fill} \\}
    
    \begin{center}
    \adjustimage{max size={0.9\linewidth}{0.9\paperheight}}{output_68_3.png}
    \end{center}
    { \hspace*{\fill} \\}
    

    % Add a bibliography block to the postdoc
    
    
    
    \end{document}
